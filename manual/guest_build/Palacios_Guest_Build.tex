\documentclass{article}[11pt]

\usepackage{amsmath}
\usepackage{amssymb}
\usepackage{verbatim}
\usepackage{epsf}
\usepackage{graphicx}
\usepackage{hyperref}

\def\colfigsize{\epsfxsize=5in}

\pdfpagewidth 8.5in
\pdfpageheight 11.0in

\setlength\topmargin{0in}
\setlength\evensidemargin{0in}
\setlength\oddsidemargin{0in}
\setlength\textheight{8.0in}
\setlength\textwidth{6.5in}

\title{Building a bootable guest image for Palacios and Kitten}

\begin{document}
\maketitle

\section{Getting the guest image build tools}

In order to build the bootable guest iso image, we need to build a Linux kernel
image from source and an initial file system containing a set of useful binary
files which will be described in the following text. We will use a new directory
for demonstration; the root directory for the following examples is
\verb+test/+:

\begin{verbatim}
mkdir test/
cd test/
\end{verbatim}

\vspace{10pt}
\noindent
There are a set of tools and sources that are useful for the guest image
building procedure. You can checkout these resources from our git repositories;
to check them out to your local \verb+test/+ directory use the following
commands: 

\begin{verbatim}
git clone http://hornet.cs.northwestern.edu:9005/busybox
git clone http://hornet.cs.northwestern.edu:9005/initrd
git clone http://hornet.cs.northwestern.edu:9005/linux-2.6.30.y
\end{verbatim}

\section{Building the ramdisk filesystem}

% Introductory text explaining why a ramdisk filesystem is necessary, and a
% small blurb about what it is. Mostly this is necessary because the
% introduction said that the "useful binary files" would be described.

Jack has made an initial ramdisk system that you can leverage. The file is
temporarily in the directory
\verb|/home/jarusl/initrd/disks/v3vee_initramfs.tar.gz| on the
newskysaw machine. If you require a custom initial ramdisk, copy the directories
and files that you require into the \verb+initramfs+/ directory. For minimal
device support, copy theses devices into the \verb+initramfs/dev/+ directory:
console, ram, null, tty (you probably need root privilege to copy and make the
device files).



\pagebreak
\begin{figure}[ht]
  \begin{center}
    \colfigsize\epsffile{busyboxConf1.eps}
    \caption{BusyBox configuration}
    \label{fig:busyboxcf1}
  \end{center}
\end{figure}

\begin{figure}[h]
  \begin{center}
    \colfigsize\epsffile{busyboxConf2.eps}
  \end{center}
  \caption{BusyBox configuration}
  \label{fig:busyboxcf2}
\end{figure}

\section{Configuring and installing BusyBox tools}

BusyBox is a software application released as Free software under the GNU GPL
that provides many standard Unix tools. BusyBox combines tiny versions of many
common UNIX utilities into a single small executable. For more details on
BusyBox, visiting \url{http://busybox.net}. To configure BusyBox, in the
\verb+busybox/+ directory, type the following:

\begin{verbatim}
make menuconfig
\end{verbatim}

or

\begin{verbatim}
make xconfig (X version)
\end{verbatim}

\vspace{10pt}
\noindent
You can add the tools you need into the guest image. There are two required
configuration options: in ``\verb|BusyBox settings->Build Options|", check the 
``\verb|Build BusyBox as a static binary|" option, and in
``\verb|BusyBox settings->Installation Options|", set the
``\verb|Busybox installation prefix|" to the path of your \verb|initramfs|
directory, as shown in figure \ref{fig:busyboxcf2}. After you finish configuring
BusyBox, save your configuration and quit the window. Then, to make the BusyBox
tools, type the following:

\begin{verbatim}
make
\end{verbatim}
Install the tools to your initial ramdisk directory:
\begin{verbatim}
make install
\end{verbatim}

\begin{figure}[ht]
  \begin{center}
    \colfigsize\epsffile{linuxConf.eps}
  \end{center}
  \caption{Linux Kernel configuration}
  \label{fig:linuxcf}
\end{figure}


\section{Configuring and compiling the Linux kernel}

Change to the \verb|linux-2.6.30.y/| directory (or whatever your Linux kernel
source directory is named) and type the following:

\begin{verbatim}
make menuconfig
\end{verbatim}
or
\begin{verbatim}
make xconfig (X version)
\end{verbatim}

\vspace{10pt}
\noindent
Configure the kernel to meet your requirements. There is a custom configuration
file \verb|jrl-default-config| which is configured with minimal kernel options
(all unnecessary options are removed to keep the guest booting process fast).
For more on configuring and compiling Linux kernel images, check online.

\vspace{5pt}
\noindent
The kernel must be configured with the initial ramdisk file system directory
(e.g. \verb|initrd/initramfs|): in the ``\verb|General setup|" menu under
option
``\verb|Initial RAM filesystem and RAM disk support|" set the
``\verb|Initramfs source file(s)|" option to the path of your \verb|initramfs|
directory as shown in figure \ref{fig:linuxcf}. When you are finished
configuring the kernel, save your configuration, and type the following:
\begin{verbatim}
make
\end{verbatim}
Some blurb about where the kernel image is...


\section{Configuring guest devices}

Checkout the updated Palacios repository to \verb|palacios/|.  (You can find
instructions for checking out the Palacios and Kitten repositories at
\url{http://www.v3vee.org/palacios/}). To build the guest VM creator tool,
change to the \verb|palacios/utils/guest_creator| directory and type the
following:

\begin{verbatim}
make
\end{verbatim}

\vspace{10pt}
\noindent
You will get the \verb|build_vm| executable. The guest configuration file uses
XML. A sample configuration file is provided: \verb|default.xml|.

\vspace{5pt}
\noindent
***Various information about how to configure the VM.***

\vspace{5pt}
\noindent
Once you have configured the VM, type the following to build the full guest VM
image:
\begin{verbatim}
./build_vm myconfig.xml -o guest.iso
\end{verbatim}
where \verb+myconfig.xml+ is your guest configuration file, and \verb+guest.iso+
is the output image file that will be used to configure kitten in the next
section.




\pagebreak
\begin{figure}[h]
  \begin{center}
    \colfigsize\epsffile{kittenConf1.eps}
  \end{center}
  \caption{Kitten configuration}
  \label{fig:kittencf}
\end{figure}

\begin{figure}[h]
  \begin{center}
    \colfigsize\epsffile{kittenConf2.eps}
  \end{center}
  \caption{Kitten configuration}
  \label{fig:kittencf2}
\end{figure}


\section{Configuring and building Palacios and Kitten}
\subsection*{Configuring and building Palacios}

You can find the detailed manual of getting and building Palacios and Kitten 
from scratch in the Palacios website (\url{http://www.v3vee.org/palacios}). Here
we only give the specific requirements related to the booting guest procedure.
To configure Palacios, in the Palacios root directory (i.e. \verb+palacios/+)
type the following:

\begin{verbatim}
make config
\end{verbatim}
or
\begin{verbatim}
make xconfig
\end{verbatim}

\vspace{10pt}
\noindent
When you have configured the components you want to build into Palacios type
the following:

\begin{verbatim}
make
\end{verbatim}
or
\begin{verbatim}
make all
\end{verbatim}

\vspace{10pt}
\noindent
Once the Palacios static library has been built you can find the library file
\verb+libv3vee.a+ in the Palacios root directory.

\subsection*{Configuring and building Kitten}

To build Kitten, first configure it in as you did Palacios. Change to the
\verb+kitten/+ directory and type the following:

\begin{verbatim}
make config
\end{verbatim}
or
\begin{verbatim}
make xconfig
\end{verbatim}

\vspace{10pt}
\noindent
Under the ``\verb|Virtualization|" menu select the
``\verb|Include Palacios virtual machine monitor|" option. Set the
``\verb|Path to pre-built Palacios tree|" option to the path of your Palacios
build, and set the ``\verb|Path to guest ISO image|" option to the path
containing the guest image that was built in the Configuring guest devices
section of this manual.

\vspace{10pt}
\noindent
After configuring Kitten, to build Kitten, type the following:

\begin{verbatim}
make isoimage
\end{verbatim}

\vspace{10pt}
\noindent
This builds the bootable ISO image file with guest OS, Palacios and Kitten.
The ISO file is located in \verb+kitten/arch/x86_64/boot/image.iso+.

\pagebreak
\noindent
You have successfully created a guest CD image file that can be booted on a
machine. You can boot the file on Qemu using the following sample command:

\begin{verbatim}
/opt/vmm-tools/qemu/bin/qemu-system-x86_64 \
        -smp 1 \
        -m 2047 \
        -serial file:./serial.out \
        -cdrom kitten/arch/x86_64/boot/image.iso \
        < /dev/null
\end{verbatim}

\vspace{10pt}
\noindent
We have finished the entire procedure for building a guest image and booting it
on the Palacios VMM. For more updated details, check the Palacios website
\url{http://www.v3vee.org/palacios} and Kitten website
\url{https://software.sandia.gov/trac/kitten} regularly.

\end{document}